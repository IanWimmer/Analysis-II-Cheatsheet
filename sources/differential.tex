% Tex root = ../cheatsheet.tex
\section{Differenzielle Analysis in R\textsuperscript{n}}
\subsection{Parzielle Ableitungen}
\definition{3.3.11}{Gradient und Divergent}
  \textbf{Gradient}: Wenn für die Funktion $f:U\rightarrow \mathbb R$ alle partiellen Ableitungen existieren für $x_0\in U$, dann ist der Vektor $$\begin{pmatrix}\partial_{x_1}f(x_0)\\\vdots\\\partial_{x_n}f(x_0)\end{pmatrix}$$
    \textbf{Divergent} Wenn für eine Funktion $f=\{f_1,...,f_m\}:U\rightarrow\mathbb R^m$ alle partiellen ableitungen für alle $f_i$ bei $x_0\in U$ existieren, ist der Divergent die Trace der Jakobimatrix $$div(f)(x_0)=Tr(J_f(x_0))$$
\subsection{Das Differential}
\definition{3.4.2}{Differenzierbarkeit}
  Wenn \(U\in\mathbb R^n\) eine offene Menge, \(f:U\rightarrow R^m\) eine Funktion und $A: \mathbb R^n\rightarrow\mathbb R^m$ eine affine Abbildung ist, dann ist $f$ bei $x_0\in U$ differenzierbar mit Differenzial A, falls:
  \[\lim\limits_{\substack{x\rightarrow x_0 \\ x\neq x_0}}\frac{f(x)-f(x_0)-A(x-x_0)}{||x-x_0||}=0\]\\
\proposition{3.4.4}{Eigenschaften von differenzierbaren Funktionen}
  Wenn \(U\in\mathbb R^n\) eine offene Menge, \(f:U\rightarrow R^m\) eine differenzierbare Funktion dann gilt:
  \begin{enumerate}
    \item Die Funktion $f$ ist stetig auf $U$
    \item Für die Funktion $f=[f_1,...,f_m]$ existieren alle \(\partial_{x_j}f_i\) mit \(1\leq j \leq n, 1\leq i\leq m\)
  \end{enumerate}
\proposition{3.4.6}{Differenzierbarkeit bei Funktionsoperationen} 
  \(U\in\mathbb R^n\) offen, \(f,g:U\rightarrow\mathbb R^m\) differenzierbar:
  \begin{enumerate}
    \item \(f+g\) ist differenzierbar und \\ $d(f+g)(x_0)=df(x_0)+dg(x_0)$
    \item Falls \(m=1: f\cdot g\) differenzierbar.
    \item Falls \(m=1, g\neq0:\frac f g\) differenzierbar.
  \end{enumerate}
\proposition{3.4.7}{Differenzial von elementaren Funktionen}
\proposition{3.4.9}{Kettenregel} \(U\in\mathbb R^n\) und \(V\in\mathbb R^m\) offen, \(f:U\rightarrow V, g:V\rightarrow\mathbb R^p\) differenzierbar.\\
\textbf{Funktionen:} 
Dann ist $g\circ f$ differenzierbar und $d(g\circ f)(x_0)=dg(f(x_0))\circ df(x_0)$.\\
\textbf{Jakobi Matrizen:}
\(J_{g\circ f}(x_0)=J_g(f(x_0)\cdot J_f(x_0)\).\\
\textbf{Gradienten:}
\(\Delta_{g\circ f}=J{g\circ f}^T, \Delta_g=J_g^T\) also
\(\Delta_{g\circ f}(x_0)=J_f(x_0)^T\cdot\Delta_g(f(x_0))\).\\
\definition{3.4.11}{Der Tangentialraum}
  \(U\in\mathbb{R}^n\) offen, \(f: U\rightarrow\mathbb{R}^m\) differenzierbar, \(x_0\in U\), \(A=df(x_0)\). Der Tangentialraum bei \(x_0\) des Graphen von \(f\) ist der Graph von \(g(x)=f(x_0)+A(x-x_0)\), also \(T=\{(x,g(x))\mid x\in\mathbb{R}^n\}\subseteq\mathbb{R}^n\times\mathbb{R}^m\).\\
\definition{3.4.13}{Richtungsableitung}
  \(U\in\mathbb{R}^n\) offen, \(f:U\rightarrow \mathbb{R}^m, v\in\mathbb{R}^n\setminus\{0\}\), \(x_0\in U\). Die Richtungsableitung von \(f\) bei \(x_o\) in richtung \(v\) ist \[D_v f(x_o) := J_g(0)=\begin{pmatrix}g'(0)_1\\\vdots\\g'(0)_m\end{pmatrix}\in\mathbb R^m\] für die Hilfsfunktion \(g : \{t\in\mathbb R\mid x_0 + tv\in U\}\rightarrow \mathbb R ^m\) \(g(t)=f(x_0+tv)\)\\
\proposition{3.4.15}{Richtungsableitung von differenzierbaren Funktionen Berechnen}
  \(U\in\mathbb R^n\) offen, \(f:U\rightarrow\mathbb R^m\) differenzierbar, \(v\in\mathbb R^m\setminus\{0\}, x_0\in U\). \\\(\implies D_vf(x_0)=df(x_0)(v)=J_f(x_0)\cdot v\) was auch bedeutet, dass die Richtungsableitung linear vom Richtungsvektor abhängen.\\
  \(\implies D_{\lambda_1v_1+\lambda_2v_2}=\lambda_1D_{v_1}f(x_0)+\lambda_2D_{v_2}f(x_0)\)\\
  \example{3.4.17}{Richtungsableitung von allgemeinen stetigen Funktionen berechnen.}
  \(D_vf(x_0)=\lim\limits_{t\rightarrow0}\frac{f(x_0+tv)-f(x_0)}{t}\) Sollte die
  daraus resultierende Funktion nicht linear von \(v\) abhängig sein, so ist
  \(f\) nicht differenzierbar.
\subsection{Höhere Ableitungen}
  \definition{3.5.1}{C Notation}
    \(U\subseteq\mathbb R^n\) offen.\\
    \(C^0(U,\mathbb R^m):=\{f:U\rightarrow\mathbb R^m\mid f \text{ stetig}\}\)\\
    \(C^k(U,\mathbb R^m):=\\
    \begin{array}{ll}\{f:U\rightarrow\mathbb R^m\mid \forall i,j:\partial_jf_i\in C^{k-1}\}\\
    \{f:U\rightarrow\mathbb R^m\mid\text{alle }\partial_{j_i}...\partial{j_k}f_i\in C^0(U,\mathbb R^m\}\\=\text{k-mal stetig differenzierbar}\end{array}\)
    \(C^\infty(U,\mathbb R^m):=\bigcup\limits_{k=0}^\infty C^k(U,\mathbb
    R^m)\)\\
    \(=\text{Beliebig oft differenzierbar bzw. glatt.}\)\\ 
  \example{3.5.2}{Nützliche C Regeln}
    \(*\) \textbf{Polynome} mit n Variablen sind in \(C^\infty(\mathbb R^n,\mathbb R)\).\\
    \(*\) \(f\in C^k\iff f_1,...,f_m\in C^k\)\\
    \(*\) \(C^k\) ist ein \textbf{Vektorraum}\\
    \(*\) Für \(k\neq 0\) ist \(\partial_j: C^k(U,\mathbb R)\rightarrow
    C^{k-1}(U, \mathbb R)\)
    \(*\) \(C^k(U,\mathbb R)\) ist abgeschlossen unter \textbf{Produkten} und
    \textbf{Summen}.
    (sofern diese Definiert sind).
    \(*\) Eine \textbf{Verknüpfung} von \(C^k\) Funktionen ist wieder \(C^k\).\\
  \proposition{3.5.4}{Satz von Schwarz}
    \(U\in\mathbb R^n\) offen, \(f\in C^2(U,\mathbb R)\). Dann gilt:
    \(\partial_i\partial_jf=\partial_j\partial_if\).
    Im Allgemeinen wenn \(f\in C^k\) dann lassen sich \(k\) parzielle
    Ableitungen beliebig vertauschen.\\
  \definition{3.5.9}{Die Hessische}
    \(U\in R^n\) offen, \(f:U\rightarrow \mathbb R, x_0\in U\).
    Die Hessische von \(f\) bei \(x_0\) ist die quadratische \(n\times
    n\)-Matrix
    \[H_f(x_0)=(\partial_i\partial_j f(x_0))_{\substack{1\leq i\leq n\\ 1\leq
    j\leq n}}\]
    Nach dem Satz von Schwarz ist \(H\) symmetrisch, falls \(f\in C^2(U,\mathbb
    R)\).\\

    
    



