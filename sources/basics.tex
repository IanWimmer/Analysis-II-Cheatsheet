% Tex root = ../cheatsheet.tex
\newcounter{basicChapter}
\newcommand{\basicdef}[2]{\definition{1.{#2}.\arabic{basicChapter}}{#1}\stepcounter{basicChapter}}
\section{Basics}
\subsection{Lineare Algebra}
  \basicdef{Norm}{1} Für $u \in\mathbf R^n, ||u|| = \sqrt[2]{u_1^2+...+u_n^2}$\\
  \basicdef{Definite Matrizen}{1}
  $A\in \mathbb R^{n\times n}$ heist...\\
  ...\textbf{positiv definit} falls $\forall v\in\mathbb R^n\setminus\{0\}: v^TAv>0$\\
  ...\textbf{positiv semidefinit} falls $\forall v\in\mathbb R^n\setminus\{0\}: v^TAv\leq0$\\
  ...\textbf{negativ definit} falls $\forall v\in\mathbb R^n\setminus\{0\}: v^TAv<0$\\
  ...\textbf{negativ semidefinit} falls $\forall v\in\mathbb R^n\setminus\{0\}: v^TAv\geq0$\\
  ...\textbf{indefinit} falls es $v,w$ gibt mit $v^TAv>0 \land w^TAw<0$\\
  Für die eigenwerte $\lambda$ von $A$ gilt:\\
  \begin{tabular}{ll}
    A pos. def.       & $\iff\forall\lambda:\lambda>0$\\
    A pos. semidef.   & $\iff\forall\lambda:\lambda\geq0$\\
    A neg. def.       & $\iff\forall\lambda:\lambda<0$\\
    A neg. semidef.   & $\iff\forall\lambda:\lambda\leq0$\\
    A indef.          & $\iff$ A hat pos. und neg. eigenwerte.\\
    $\det(A)\neq0$    & $\implies\forall\lambda:\lambda\neq0$\\
  \end{tabular}
  \subsection{Notation}
  \basicdef{Landau Notation}{2}
    \(U\in\mathbb R^n, h:U\rightarrow\mathbb R, y\in U\)\\
    \(*\) \(o(h)=\{f:U\rightarrow\mathbb R\mid \lim\limits_{\substack{x\rightarrow y\\x\neq y}}\frac{f(x)}{h(x)}=0\}\)\\
    \(*\) \(f=o(h):=f\in o(h)\)\\
    \(*\) \(o(f)=o(h):=o(f)\in o(h)\)\\
    \(*\) \(f=o(1)\iff \lim\limits_{x\rightarrow y}f(x)=0\)\\
    \(*\) \(\lambda o(h) +\mu o(h) = o(h)\;\; \forall \lambda,\mu\in\mathbb R\)\\
    \(*\) \(g\cdot o(h)=o(gh) = o(g)\cdot o(h)\)\\
    \(*\) \(o(h^d)=o(h^e)\;\forall e\leq d\)
    \(*\) Für Monome \(p\) in \(x_i-y_i\) von Grad \(d\): \(p=o(||x-y||^e)\;\forall e\leq d\) \& \(o(p)=o(||x-y||^d)\)
\subsection{Methoden}
  \basicdef{Koeffizientenvergleich}{3}
    Zwei Polynome sind genau dann gleich, wenn ihre Koeffizienten
    übereinstimmen.
    $Q(x) = P(x) \iff \deg(Q)=\deg(P)=I \land \forall i, 0 \leq i \leq I: q_i = p_i$
    Wenn wir unbekannte in den koeffizienten haben können wir damit ein
    Gleichungsystem machen.
